\documentclass[12pt,reqno]{amsart}
\usepackage[dvipsnames,svgnames]{xcolor}
\usepackage{tikz}
\usepackage[margin=1.2cm, top=0.5cm, bottom=0.5cm,a3paper]{geometry}
 \usepackage[pdfborder={0 0 0}, citecolor=blue, urlcolor={Aquamarine!50!black},
  linkcolor=orange, colorlinks=true, bookmarksopen=true]{hyperref}
\newcommand\hs[1]{\hspace{#1 cm}}
\newcommand\green[1]{\textcolor{green!50!black}{#1}}
\usetikzlibrary{external,matrix, automata,trees,positioning,shadows,arrows,shapes.geometric,shapes.multipart,trees,calc, fit, decorations.pathmorphing,decorations.pathreplacing}
\usepackage{soul}
\usepackage{setspace}
\setstretch{1.2}
\newcommand{\vv}{ \vspace{2mm} \\}

\begin{document}
\setul{1pt}{.4pt}
\phantom{123}\\ 
{\Large {\textsc{Syllabus}}}\vv
\ul{Course:} ECON 510 Section 002. Advanced Microeconomic Theory. 3 Credits.\vv
\ul{Course Description:} \begin{quote} In this course we will study decision-making  in a dynamic environments. which are often neglected in most undergraduate courses. In particular, we shall cover the following topics: optimization techniques, recursive preferences, hyperbolic preferences, inter-temporal consumption,  general equilibrium under complete and incomplete markets, search theory, overlapping generations, and dynamic games.  \end{quote}
\ul{Requisites:} Prerequisite, ECON 410 with a grade of C or better.\vv
\ul{Grading status:} Letter grade.\vv
\begin{minipage}[r]{10cm}
\ul{Instructor:} \\
 S\'ergio O. Parreiras \\
Gardner Hall, 200B\\
sergiop@unc.edu,
\end{minipage}
\begin{minipage}[r]{6cm}
\ul{Class Schedule:} \\
TR 2 pm - 3:15 pm\\
Gardner Hall, Rm 308\\ \\
\ul{Office Hours (OH):} \\
over ZOOM on Mon/Wed:\\ 9am-10am
by Canvas appt. 
\end{minipage}  \vspace{3mm}\\
\ul{Communication: }\\
The Canvas website is our primary main of communication. All assignments, readings,  grades, podcast links, and other resources will be posted on Canvas.  Please, use  Canvas$>$Pages$>$ Sign-up to schedule an office hours (OH) meeting. The sign-up opens seven days before and closes 24 hours before respective OH. You must sign with SSO login option at unc.zoom.us for OH meetings. Do not hesitate to send a Canvas message to schedule meetings \textcolor{red}{outside} regular OH if your schedule conflicts with the regular OH.  \vspace{3mm}\\
\ul{Prerequisites:} ECON 410 with a grade of C or better.  \vspace{3mm}\\
\ul{Required textbook:}\\ There is no required textbook. The Lecture Notes as well as books and articles that the with  free access thru the University Libraries will be posted/linked in the corresponding Canvas module. \\

\ul{Required software/apps:}\\ 1). Mathematica which can be ordered free of any charges at \href{https://software.unc.edu}{https://software.unc.edu}.\\ No prior knowledge of Mathematica is required for one to succeed in this course but Mathematica  will be used on an everyday basis so please make sure to install it on your  laptop as soon as possible.\\
2) Zoom which shall  be used in some classes so you can share your screen with the entire class.\\
3) Overleaf. Please, open a free individual account at \href{}{Overleaf.com}. You will use Overleaf to typeset your research assignments using the \LaTeX  markup. \vv
\ul{Learning Objectives:}\\
The course main goal is to provide tools to enable you to: construct models of micro-economic behavior in dynamic environments; identify their (built-in) limitations and; think critically about how to apply them to real-life. We will also write a research essay/proposal/project aimed at the learning outcomes described by  the \href{https://curricula.unc.edu/curriculum-proposals/cim/ideas-in-action-slos-recurring-capacities/}{IDEAS in action (Research)} curricula.
\vspace{3mm}\\

\ul{Evaluation:\\}
Course grade weights: midterms (40\%), problem sets and quizzes (20\%), final exam (20\%),  \underline{active} participation (5\%), research assignments  (15\%).
\newpage \phantom{1}\\
\ul{Exam dates:\\} Midterm 1: February 20th. Midterm 2: April 1st. Final exam:   May 1st at \textcolor{red}{noon} (notice that the time differs from class time!). \vv
\ul{Class environment:\\}
During this course, we may employ additional material from podcasts, video, or literature to discuss economics related issues. Sometimes, you may find the political or religious views; or the
profanity contained in the additional material offensive or
objectionable and, you may feel uncomfortable. I will not endorse or advocate any particular political views but as part of your university  education, it is important you engage in critical thinking and also respect different opinions expressed by your classmates. Our classroom is an inclusive environment. Your participation is critical for the success of this course. You will be expected to attend class, read the assigned readings and, be prepared to discuss and present your work in the classroom.\vv
{\large {\textsc{Policies and procedures}}}
\begin{description}
\item[Syllabus Changes]
I reserve the right to make changes to the syllabus including project due dates and test dates. These changes will be announced as early as possible.
\item [Communication]  Please, use Canvas to send me messages instead of regular email.
 All assignments should be submitted on Canvas. You will receive notifications for those classes in which Mathematica is required. 
\item [Honor Code] All students are expected to follow the guidelines of the UNC honor code. In particular, students are expected to refrain from ``lying, cheating, or stealing'' in the academic context. If you are unsure about which actions violate the honor code, please see me or consult \href{https://catalog.unc.edu/policies-procedures/honor-code/}{https://catalog.unc.edu/policies-procedures/honor-code/}.
\item [Exams and grades]  \phantom{12}\\
If you are eligible to take exams with Accessibility Resources and Service. Please schedule your exam using their hub, \href{https://ars.unc.edu/}{https://ars.unc.edu/}, and please notify me as soon as possible.\\
 Any final-exam rescheduling request, for those with more than three final exams within a 24 hours period, must be received no later than our first midterm.\\
Exam grades are converted into scores (``curved'') accordingly to the formula: original grade plus 100 minus the maximum between the top class grade and 50. Assignments or problem sets are not ``curved''. The final course grade is computed accordingly to the following scheme:
\end{description}\vspace{5mm}
\begin{center}
\begin{tabular}{|l|l|}\hline
letter grade & minimum score \\  \hline
A & 95 \\  \hline
A-&  90 \\ \hline
B+ & 87 \\ \hline
B & 83 \\ \hline
B- & 80 \\ \hline
\end{tabular}\hspace{1cm}
\begin{tabular}{|l|l|}\hline
letter grade & minimum score \\  \hline
C+ & 77 \\ \hline
C & 73 \\ \hline
C- & 70 \\ \hline
D+ & 67 \\ \hline
D & 63 \\ \hline
\end{tabular}
\end{center}\vspace{5mm}
\begin{description}
\item[Attendance Policy]
Attendance will be recorded by the UNC check-in app,  click \href{https://unccheckin.unc.edu}{here}, to install the app. No right or privilege exists that permits a student to be absent from any class meetings, except for these University Approved Absences (UAA): 1) Authorized University activities; 2)
Disability/religious observance/pregnancy, as required by law and approved by Accessibility Resources and Service and/or the Equal Opportunity and Compliance Office. 3)
Significant health condition and/or personal/family emergency as approved by the Office of the Dean of Students, Gender Violence Service Coordinators, and/or the Equal Opportunity and Compliance Office. Eight non-UAA without any consultation with the instructor may result in a grade F in the course.
Ten total non-UAA in a semester may result in a grade F regardless of the reasons you missed class. In all instances, either UAA or non-UAA you MUST notify the instructor prior to missing class. If a student misses a class, it is the student's responsibility to find out what was covered in the class and
whether changes have been made on the schedule, and if there was any assignment given.
 \end{description}\newpage
 \begin{description}
\item[Counseling and Psychological Services]
UNC-Chapel Hill is strongly committed to addressing the mental health needs of a diverse student body. The Heels Care Network website is a place to access the many mental resources at Carolina. CAPS is the primary mental health provider for students, offering timely access to consultation and connection to clinically appropriate services. Go to their website https://caps.unc.edu/ or visit their facilities on the third floor of the Campus Health building for an initial evaluation to learn more. Students can also call CAPS 24/7 at 919-966-3658 for immediate assistance. 
\item[Title IX Resources] Any student who is impacted by discrimination, harassment, interpersonal (relationship) violence, sexual violence, sexual exploitation, or stalking is encouraged to seek resources on campus or in the community. Reports can be made online to the EOC at https://eoc.unc.edu/report-an-incident/. Please contact the University’s Title IX Coordinator (Elizabeth Hall, titleixcoordinator@unc.edu), Report and Response Coordinators in the Equal Opportunity and Compliance Office (reportandresponse@unc.edu), Counseling and Psychological Services (confidential), or the Gender Violence Services Coordinators (gvsc@unc.edu; confidential) to discuss your specific needs. Additional resources are available at safe.unc.edu. 
\item[Policy on Non-Discrimination] The University is committed to providing an inclusive and welcoming environment for all members of our community and to ensuring that educational and employment decisions are based on individuals’ abilities and qualifications. Consistent with this principle and applicable laws, the University’s Policy Statement on Non-Discrimination offers access to its educational programs and activities as well as employment terms and conditions without respect to race, color, gender, national origin, age, religion, genetic information, disability, veteran’s status, sexual orientation, gender identity or gender expression. Such a policy ensures that only relevant factors are considered and that equitable and consistent standards of conduct and performance are applied.
If you are experiencing harassment or discrimination, you can seek assistance and file a report through the Report and Response Coordinators (see contact info at safe.unc.edu) or the Equal Opportunity and Compliance Office, or online to the EOC at https://eoc.unc.edu/report-an-incident/.
\item[Diversity Statement] I value the perspectives of individuals from all backgrounds reflecting the diversity of our students. I broadly define diversity to include race, gender identity, national origin, ethnicity, religion, social class, age, sexual orientation, political background, and physical and learning ability. I strive to make this classroom an inclusive space for all students. Please let me know if there is anything I can do to improve. I appreciate suggestions.
\item[Undergraduate Testing Center] The College of Arts and Sciences provides a secure, proctored environment in which exams can be taken. The center works with instructors to proctor exams for their undergraduate students who are not registered with ARS and who do not need testing accommodations as provided by ARS. In other words, the Center provides a proctored testing environment for students who are unable to take an exam at the normally scheduled time (with pre-arrangement by your instructor). For more information, visit http://testingcenter.web.unc.edu/.  
\item[Grade Appeal Process ] If you feel you have been awarded an incorrect grade, please discuss it with me. If we cannot resolve the issue, you may talk to our departmental director of undergraduate studies or appeal the grade through a formal university process based on arithmetic/clerical error, arbitrariness, discrimination, harassment, or personal malice. To learn more, go to the Academic Advising Program website.
\end{description}\newpage
\setstretch{1.2}
\hspace{5cm}\large {\textsc{Tentative coverage}}\\
\tikzset{ms1/.style={row #1 column 1/.append style={execute at begin node=Week \space \the\numexpr#1-1}}}
\tikzset{ms2/.style={row #1 column 1/.append style={execute at begin node=Week \space \the\numexpr#1+7}}}
Abbreviations: Problem set (PS); Quiz (Q); To be announced (TBA) \\
\begin{tikzpicture}
\matrix[
matrix of nodes,
column sep=-\pgflinewidth,
row sep=-\pgflinewidth,
execute at empty cell={\node{\phantom{$\cdot$}};},
    nodes={draw, align=left, anchor=north west,
text depth=0.9cm,
text height=0.9cm,
minimum width=8cm},
text width=8cm,
ms1/.list={2,3,4,5,6,7,8,9,10,11,12,13,14,15,16,17,18}]
{ &Tuesday &Thursday  \\
: Mathematica  &  & 1/9: Syllabus, Mathematica tutorial \\ 
:  Optimization & 1/14: Mathematica tutorial and Q 1 & 1/16: Micro Theory methodology\\
:  & 1/21:  Optimization (equality constraints) and  Q 2 &  1/23: Optimization, PS 01 due\\
:  Dynamic Preferences:   & 1/28:  & 1/30: Exponential and hyperbolic preferences, PS 02 due\\
: & 2/4: Recursive Preferences  & 2/6: Optimization (inequality constraints), PS 03 due\\
:    & 2/11: Optimization (inequality constraints) &  2/13: Comparative Statics and Envelope theorem, PS 04 due\\
:    & 2/18:  The Value Function  & 2/20: Midterm 1\\
: General Equilibrium   & 2/25:  Time consistency &2/27: Complete Markets, PS 05 due\\
:   &3/4: Risk and Uncertainty & 3/6: Incomplete Markets\\
:  Spring Break &3/11: Spring Break & 3/13: Spring Break \\
:  OLG  &3/18: Incomplete Markets & 3/20: Overlapping generations, PS 06 due\\
:  Search   & 3/25: Overlapping generations & 3/27: Search Theory, PS 07 due\\
:  Dynamic Games  & 4/1: Midterm 2 & 4/3: Repeated games, PS 08 due\\
:  Dynamic Games & 4/8: Repeated games & 4/9: Markov games, PS 09 due\\
:   & 4/15: Markov games & 4/17: Well-being day\\
:  & 4/22: PS 10 due & 4/24: Research paper due\\
:  & 4/28:  Review &  5/1 at \textcolor{red}{noon}: Final Exam \\};
\end{tikzpicture}

\end{document}

